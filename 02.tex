\setcounter{section}{5}

\section{Інтегрування полів: потік векторного поля. Теорема Гауса для електричного поля}
%%%%%%%%%%%%%%%%%%%%%%%%%%%%%%%%%%%%%%%
\begin{problem}{1}
Знайти напруженість поля рівномірно зарядженої площини з поверхневою густиною заряду~$\sigma$.
\end{problem}

\begin{problem}{2}
Знайти напруженість поля двох паралельних площин. Відома поверхнева густина заряду~$\sigma$.
\end{problem}

\begin{problem}{3}
Знайти напруженість поля рівномірно зарядженої сферичної поверхні. Радіус поверхні $R$, заряд $-q$.
\end{problem}

\begin{problem}{4}
Визначити потік $\Phi$ вектора напруженості електричного поля $Е$ крізь бічну поверхню конуса, висота якого $h = 20$~см і радіус основи $r = 10$~см. На осі конуса на однакових відстанях від вершини й центра основи розміщений заряд $q = 1$~мкКл. Конус міститься у вакуумі.
\end{problem}

\begin{problem}{5}
У центрі куба міститься точковий заряд $q$. Чому дорівнює потік $\Phi$ вектора напруженості $Е$: а) крізь повну поверхню куба; б) крізь одну з його граней?
\end{problem}

\textbf{Завдання для самостійної роботи}

\begin{problem}{6}
Обчислити потік $\Phi$ вектора напруженості електричного поля $Е$ крізь бічну поверхню прямого кругового циліндра, висота якого $h = 20$~см, а радіус основи $r = 10$~см. Точковий заряд $q= 0,3$~мкКл міститься: а) на осі циліндра на середині висоти; б) у центрі основи.
\end{problem}

\begin{problem}{7}
Два нескінченних тонкостінних коаксіальних циліндри радіусів $R_1 = 5$~см і $R_2 = 10$~см рівномірно заряджені з поверхневими густинами зарядів $\sigma_1= 10$~нКл/м$^2$ і $\sigma_2= -3$~нКл/м$^2$. Простір між циліндрами заповнений повітрям.
Визначити модуль $Е$ напруженості поля в точках, що містяться на відстанях $r_1 = 2$~см, $r_2 =6$~см, $r_3 = 15$~см від осі циліндрів.
\end{problem}

\begin{problem}{8}
Усередині однорідно зарядженої кулі з об'ємною густиною заряду $\sigma$ міститься сферична порожнина, центр якої зміщений відносно центра кулі на відстань $r_0$. Визначити напруженість $Е$~поля всередині порожнини. Якою вона буде при $r_0 = 0$? 

\nb Скористатися принципом суперпозиції полів
позитивно об'ємнo зарядженої кулі і негативно об'ємно зарядженої кулі з радіусом порожнини.
\end{problem}