\setcounter{section}{5}

\section{Інтегрування полів: потік векторного поля. Теорема Гауса для електричного поля}
%%%%%%%%%%%%%%%%%%%%%%%%%%%%%%%%%%%%%%%

\hrulefill
\begin{theorem}[Теорема Гауса для електричного поля]
$\oint\limits_S \Vec{E} \cdot \Vec{n} \d S = \frac{Q}{\varepsilon_{0}} = \frac{1}{\varepsilon_{0}} \oint_V \rho \d V$ \\
Потік електричного поля через довільну \textbf{замкнену} поверхню рівний сумарному електричному заряду всередині цієї поверхні, помноженому на $\frac{1}{\varepsilon_{0}}$.
\end{theorem}
\hrulefill

\begin{problem}
Знайти напруженість поля рівномірно зарядженої площини з поверхневою густиною заряду~$\sigma$.
\end{problem}
	
\renewcommand{\given}{}
\renewcommand{\question}{}
\renewcommand{\innertext}{}
\renewcommand{\imageone}{image}

\genpreambula


	
	